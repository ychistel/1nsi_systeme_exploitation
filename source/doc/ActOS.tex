\documentclass[11pt,a4paper]{article}

\usepackage{style2017}
\usepackage{hyperref}

\hypersetup{
    colorlinks =false,
    linkcolor=blue,
   linkbordercolor = 1 0 0
}
\newcounter{numexo}
\setcellgapes{1pt}

\begin{document}

\begin{Huge}
\textbf{Activité} : Système d'exploitation
\end{Huge}
\medskip
\hrule
%\begin{NSI}
%{Activité}{Système d'exploitation}
%\end{NSI}

\section{Introduction}
%En informatique, un \textbf{système d'exploitation} (souvent appelé OS de l'anglais Operating System) est un ensemble de programmes qui dirige l'utilisation des ressources d'un ordinateur par des logiciels applicatifs.
%
%Les principales fonctionnalités d'un système d'exploitation sont:
%\begin{itemize}
%\item La gestion des fichiers, leur organisation.
%\item Proposer un ensemble de logiciels de base pour l'utilisateur.
%\item La gestion des utilisateurs de la machine.
%\item La prise en charge du matériel informatique
%\item La gestion de la mémoire et des programmes.
%\end{itemize}
%\medskip
\begin{enumerate}
\item Expliquer en quelques lignes le rôle d'un système d'exploitation sur un ordinateur. \vspace{15cm}
\item Citer différents systèmes d'exploitation et dessiner leur logo.\vspace{6cm}
%\item Expliquer en 2 ou 3 lignes ce qu'est Windows \vspace{2cm}
%\item Quelle est la différence entre un système d'exploitation propriétaire et un système d'exploitation libre ? 
\end{enumerate}

\newpage
\section{Le système de gestion de fichiers}

Un système d'exploitation gère les fichiers et les dossiers des utilisateurs, des programmes et du système lui-même. Il s'occupe de pouvoir les lire et les écrire en mémoire vive et sur les différents espaces de stockage comme le disque dur quand on lui demande. Un ordinateur contient des milliers de fichiers. L'objectif est de comprendre comment ils sont organisés sur la machine.

\begin{enumerate}
\item Comment sont organisés les fichiers par le système d'exploitation Windows ? \vspace{3cm}

\item Représenter l'organisation des fichiers de votre PC par le système d'exploitation Windows. \vspace{8cm}
\item Que contiennent les dossiers \textbf{Programmes}, \textbf{Utilisateurs} et \textbf{Windows} ? \vspace{4cm}


\item Dans tout système d'exploitation, le nom d'un fichier se compose de deux parties séparées par un point. Avant le point, c'est le nom du fichier, après le point c'est son extension.

\begin{enumerate}
\item Créer, dans votre espace personnel, un fichier texte avec le bloc notes de windows contenant un court texte puis enregistrer votre fichier sous le nom \textsf{extension}. Quelle est l'extension de votre fichier ? \vspace{1cm}

\item Remplacer l'extension de votre fichier par l'extension \textsf{png}. Quelle est l'application proposée pour ouvrir votre fichier (faire un clic droit sur le nom de votre fichier). \vspace{1cm}

\item À quoi sert donc l'extension d'un fichier ? \vspace{1cm}
\end{enumerate}
\end{enumerate}


\newpage
\section{La prise en charge du matériel}

Un système d'exploitation prend en charge les différents périphériques d'un ordinateur. Dans le cas de Windows, différents programmes permettent au système d'exploitation d'échanger avec le matériel.

\begin{enumerate}
\item Comment appelle-t-on les programmes permettant d'utiliser les périphériques de l'ordinateur ? \vspace{3cm}


\item Le système d'exploitation Windows dispose d'un programme \textsf{msinfo32.exe} qui donne des informations sur le matériel.

\begin{enumerate}
\item Quel est le processeur de votre machine ? Combien de coeurs a-t-il ? Quelle est la fréquence d'horloge ?  \vspace{3cm}

\item Quelle est la taille de la mémoire vive (RAM) ? \vspace{3cm}

\item Quelle est la carte graphique utilisée ? A-t-elle une mémoire dédiée ? \vspace{3cm}

\item Quelle est la capacité du disque dur de votre machine ? \vspace{3cm}

\end{enumerate}
\end{enumerate}

\newpage
\section{La gestion des utilisateurs}
Un système d'exploitation gère les utilisateurs de la machine. Lorsqu'un utilisateur s'identifie, il interdit l'accès aux fichiers et dossiers qui ne lui appartiennent pas. C'est ce qu'on appelle les \textbf{permissions utilisateurs} ou \textbf{droits des fichiers}.

Pour connaitre les permissions, on peut sélectionner un fichier, faire un clic doit et ouvrir la fenêtre propriétés puis l'onglet sécurité.

\begin{enumerate}
\item Quels sont les différents types d'utilisateurs gérés par Windows ? \vspace{3cm}

\item Quelles sont les permissions que l'on peut accorder à un fichier ? \vspace{3cm}

\item Les permissions sont-elles différentes entre les utilisateurs ?\vspace{3cm}
\end{enumerate}





\end{document}

