\documentclass[11pt,a4paper]{article}

\usepackage{style2017}
\usepackage{hyperref}

\hypersetup{
    colorlinks =false,
    linkcolor=blue,
   linkbordercolor = 1 0 0
}
\newcounter{numexo}
\setcellgapes{1pt}

\begin{document}



\begin{NSI}
{Exercice}{Commandes de base linux}
\end{NSI}

Vous trouverez de l'aide sur les commande linux sur le site : \href{https://doc.ubuntu-fr.org/tutoriel/console_commandes_de_base}{\blue{Les commandes de base en console linux}}

\addtocounter{numexo}{1}
\subsection*{\Large Exercice \thenumexo }
On suppose que le répertoire personnel de l'utilisateur courant est vide. 

\begin{enumerate}
\item Décrire l'effet des commandes suivantes en supposant qu'elles sont exécutées les unes après les autres dans cet ordre.
\begin{enumerate}
\item cd ~
\item mkdir NSI
\item mkdir NSI/TP\_SHELL
\item cd NSI/TP\_SHELL
\item ls
\item cd ..
\item mkdir PYTHON
\item ls -l
\item chmod u+rwx,g-rwx,o-rwx TP\_SHELL
\end{enumerate}
\item Représenter l'arborescence de fichiers du dossier utilisateur.
\end{enumerate}


\addtocounter{numexo}{1}
\subsection*{\Large Exercice \thenumexo }

On utilisera les commandes suivantes :

\begin{itemize}
\item La commande \textsf{touch nom\_fichier} permet de créer un fichier vide en donnant un nom au fichier.

\item Le programme \textsf{nano nom\_fichier} est un éditeur de texte qui permet de visualiser et modifier un fichier.

\item La commande \textsf{rm nom\_fichier} permet de supprimer le fichier indiqué après la commande.

\item La commande \textsf{echo message} affiche le message indiqué après la commande.
\end{itemize}


\begin{enumerate}
\item Placez vous dans le répertoire TP\_SHELL et créer un fichier vide nommé \textsf{exolinux.txt}.
\item Editez le fichier \textsf{exolinux.txt} et insérer le texte suivant: "Je peux tout faire avec les commandes linux." 

Sauvegarder et quitter l'éditeur.
\item Afficher votre fichier dans la console avec la commande \textsf{cat}.
\item Exécuter la commande \textsf{echo} "bonjour". Que se passe-t-il ?
\item Exécuter la commande \textsf{echo bonjour} > \textsf{exolinux.txt}. Que se passe-t-il ?

Afficher votre fichier \textsf{exolinux.txt}. Que remarquez-vous ?
\item Exécuter la commande \textsf{echo au revoir} $>>$ \textsf{exolinux.txt}. Que se passe-t-il ?

Aller voir votre fichier \textsf{exolinux.txt}.
\item Déplacer le fichier \textsf{exolinux.txt} dans le dossier Documents.
\item Supprimer le fichier \textsf{exolinux.txt}.
\end{enumerate}



%\addtocounter{numexo}{1}
%\subsection*{\Large Exercice \thenumexo }
%%\subsubsection*{Préambule:}
%Le groupe \textbf{sudo} permet à ses membres d'exécuter des commandes réservées au super utilisateur \textbf{root}.
%
%Pour exécuter des commandes super utilisateur, il suffit de faire précéder la commande par le mot clé \textbf{sudo}.\smallskip
%
%L'utilisateur \textbf{pi} appartient au groupe des super utilisateurs.\medskip
%
%La commande \textbf{adduser} \textit{user} permet d'ajouter un utilisateur.\bigskip



%\begin{enumerate}
%\item Ouvrez un terminal et relever les identifiants numériques (UID et GID) de l'utilisateur \textbf{pi}.
%\item Ajouter l'utilisateur \textbf{toto} avec le mot de passe \textbf{nsi}. Relever les identifiants UID et GID de \textbf{toto}.
%
%\item Une fois l'utilisateur \textbf{toto} créé, déconnectez l'utilisateur \textbf{pi} et connectez-vous en tant qu'utilisateur \textbf{toto}.
%\item Ouvrez le gestionnaire de fichier:
%\begin{enumerate}
%\item Pouvez-vous vous rendre dans le répertoire de \textbf{pi} ? Pourquoi ?
%\item Pouvez-vous lire un fichier dans le répertoire de \textbf{pi} ? Pourquoi ? Le modifier ?
%\item Pouvez-vous créer un fichier dans le dossier personnel de \textbf{pi} ? Pourquoi ?
%\end{enumerate}
%
%\end{enumerate}


%\addtocounter{numexo}{1}
%\subsection*{\Large Exercice \thenumexo }
%L'objectif de cet exercice est de rendre les sous-dossiers (Documents, Pictures, ...) de l'utilisateur \textbf{pi} non accessible et non visible à l'utilisateur \textbf{toto}.
%
%Pour cela, il faut modifier les droits (permissions) sur le dossier \textbf{pi}. Seul un super utilisateur peut modifier ces permissions. \medskip
%
%Effectuer les changements et contrôler que \textbf{toto} ne peut ni voir, ni ouvrir un fichier de l'utilisateur \textbf{pi}, y compris en ligne de commande.



\addtocounter{numexo}{1}
\subsection*{\Large Exercice \thenumexo}
\textbf{Cet exercice devra être réalisé sur Windows et sur Linux.}\bigskip

Python dispose d'un module qui permet d'exécuter des commandes au niveau du système d'exploitation. 

Ce module est \textsf{os}.\medskip

La méthode \textsf{system} de ce module permet d'exécuter une commande. 

Par exemple:

\begin{itemize}
\item sur windows, \textbf{os.system("C:/chemin/vers/programme")} ouvre le bloc notes;
\item sur Linux, \textsf{os.system("/chemin/vers/programme")} ouvre l'éditeur de texte.
\end{itemize}  

On récupère le prompt de l'interpréteur python seulement en fermant l'application .\medskip

La méthode \textsf{startfile} permet l'ouverture d'un fichier. 

Elle prend en argument le chemin complet du fichier à ouvrir.\medskip

D'autres méthodes utiles sont données ci-après (attention aux arguments):
\begin{itemize}
\item \textsf{name} renvoie le nom de l'OS.
\item \textsf{getcwd()} renvoie le répertoire courant.
\item \textsf{listdir()} renvoie le contenu du répertoire courant.
\item \textsf{mkdir} crée un répertoire.
\item \textsf{chdir} change de répertoire courant
\item \textsf{remove} supprime un fichier.
\item \textsf{rmdir} supprime un répertoire.
\end{itemize}\bigskip

Ouvrir l'interpréteur python et importer le module \textsf{os}. 
Toutes les actions demandées ci-après se font \textsf{exclusivement} en python.\medskip

\begin{enumerate}
\item Afficher le nom du système d'exploitation.
\item Quel est le répertoire courant ?
\item Lister le contenu du répertoire courant. 
\begin{enumerate}
\item Combien y a-t-il de dossiers et fichiers dans le répertoire courant ?
\item Un fichier et un dossier sont cachés si le nom commence par un point. 

Pouvez-vous trouver combien il y en a (en python bien sur)?
\end{enumerate}
\item Créer un répertoire \textsf{titi} dans le répertoire courant puis placez-vous dans le répertoire \textsf{titi}.
\item Créer un fichier texte \textsf{test.txt} contenant un message de bienvenue.
%\item Ouvrir un navigateur et afficher la page d'accueil du site du lycée.
\end{enumerate}



\end{document}


